\documentclass[12pt, twoside]{report}
\usepackage{graphicx} % Required for inserting images
\usepackage[a4paper, total={6in, 9in}]{geometry}
\graphicspath{ {images} }
\usepackage{datetime, xcolor, caption, subcaption, multicol}
\usepackage{amsthm, thmtools, amsfonts, mathtools, fancyhdr, hyperref}

\pagestyle{fancy}




\newdateformat{monthyeardate}{\monthname[\THEMONTH] \THEYEAR}

\title{}
\author{Riddhiman}
\date{\monthyeardate\today}

\rhead{\includegraphics[width=1.25cm]{images/bits-logo-color}}
\lhead{Assignment-I}

\setlength\headheight{40pt} 


\begin{document}

\begin{titlepage}

	\begin{center}

		\vspace*{1cm}
		\Huge
		\textbf{Assignment-I}

		\vspace{0.5cm}

		\LARGE
		\textbf{BITS F-112 Technical Report Writing}

		\vspace{0.8cm}

		\Large	
		Group No. 6

		\vspace{0.6cm}

		\normalsize
		\begingroup
		\setlength{\tabcolsep}{10pt}
		\renewcommand{\arraystretch}{1.8}
		\begin{tabular} { || c | c | c || }
			\hline
			\textbf{Name} & \textbf{Id} & \textbf{Email} \\ [0.5ex]
			\hline \hline

			\hline  
			Shivansh Vashisth & 2023A1PS0107G & f20230107@goa.bits-pilani.ac.in \\  
			Riddhiman Sengupta & 2023A1PS0108G & f20230108@goa.bits-pilani.ac.in \\
			Aryaman Karoliwal & 2023A1PS0109G & f20230109@goa.bits-pilani.ac.in \\
			Aditya Biyani & 2023A1PS0112G & f20230112@goa.bits-pilani.ac.in \\
			Kheelendra Kumar Dewangan & 2023A1PS0113G & f20230113@goa.bits-pilani.ac.in \\
			[1ex]
			\hline
		\end{tabular}
		\endgroup
		

		\vfill

		\includegraphics[width=0.4\textwidth]{bits-logo-color}

		\Large
		BITS Pilani, K. K. Birla Goa Campus \\ 
	\end{center}

\end{titlepage}



% input contributions page

\begin{center}
	\section*{Contributions}%
	\begingroup
		\setlength{\tabcolsep}{10pt}
		\renewcommand{\arraystretch}{1.8}
		\begin{tabular} { ||c|p{81mm}|| }
			\hline
			\textbf{Name} & \ \ \ \ \ \ \ \ \ \ \ \ \ \ \ \ \ \ \ \  \textbf{Contributions} \\ [0.5ex]
			\hline \hline

			\hline  
			Shivansh Vashisth & Contributed \textit{Interest} \\ 
			\hline
			Riddhiman Sengupta & Contributed \textit{Research Problem} and compilation \\
			\hline
			Aryaman Karoliwal & Contributed \textit{Research Questions}, \textit{Significance} and compilation \\
			\hline
			Aditya Biyani & Contributed \textit{Research Problem} \\
			\hline
			Kheelendra Kumar Dewangan & Contributed \textit{Significance} and \textit{Research Questions} \\
			[1ex]
			\hline
		\end{tabular}
		\endgroup
\end{center}


\newpage 

\begin{multicols}{2}
\section*{Interest}
We are interested in ``climate change and food security'' due to the profound global implications of effects of a changing climate on agricultural systems and food production. We are specifically geared towards the far-reaching consequences for local communities and ecosystems. We wish to find sustainable solutions that can mitigate the adverse impacts of climate change on food security, customized to the economic and social situations of localities. Our interests are to address the looming crisis, promote resilience, and ensure equitable access to nourishment for future generations.


\section*{Research problem}
The cited materials suggest that food security, availability, and climate change are interlinked. We also gather food security encompasses the utilization and adequacy of food, estimating the deficiency in food available to a particular sample population and larger group. \\ \\ 
However, concerns arise when rural food system transformation is hindered by lack of infrastructure, resources, knowledge, and understanding. Moreover, global warming due to uneven heating adds to their plight, along with droughts and floods, which make them unable to connect the dots regarding the larger task of ensuring food availability. \\ \\
In this research, we are interested in exploring various different steps and policies that governments and organizations can implement in local communities to bring about effective changes in the four broad dimensions of food security.


\section*{Research objective}%
This research assesses the viability of transforming food systems in different populations, considering economic, social, and human resource development, and explores mitigation frameworks for these problems. \\ 
We shall investigate the same by posing the following questions:
\begin{itemize}
	\item What are the range of direct or indirect effects climate change can have on the four dimensions of food security - availability, accessibility, utilization and stability in local communities?
	\item What impact would inefficiencies in local bureaucracy and regulations cause in implementation of programmes geared towards food system transformation, and how can artificial intelligence with other digital infrastructure play a role in mitigating the impact?
	\item How can local cultural cuisines of smaller geographical areas help in building specific food systems that are most optimal for that locality and will it be economically viable to create self-sustainable local “pockets” in terms of food security?
\end{itemize}

\section*{Significance of research problem}%
Climate change is causing disruption in food systems by ravaging agricultural production, thereby exacerbating the issue of food insecurity and nutritional deficiency. It could reduce the GDP of the world’s poorest by $ 30 \% $ and cause $ 300 $ million cases of malnutrition in protein and zinc deficiency (Sova, 2019). \\
Transforming food systems to adapt to climate change will open up research avenues and can help optimize food distribution, agricultural practices, crop viability, operational efficiency of local agricultural economies and many more. \\ \\
Research in transforming food systems can bring about progress in promoting global equitability in vital ``life'' infrastructure. Entrepreneurially, such research also opens up monetary avenues in the HORECA domain by creating products from locally sourced crops native to the area. \\ \\
Thus we need to shed light on this issue for actionable recommendations about how to mitigate the vulnerability of food systems in the face of climate change.

\end{multicols}

\flushright
\textbf{Word Count:} 494

\bibliographystyle{plain}
\bibliography{refs}
\nocite{gregory2005}
\nocite{sova2019}
\nocite{wheeler2013}

\end{document}

